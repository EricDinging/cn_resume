% !TEX TS-program = xelatex
% !TEX encoding = UTF-8 Unicode
% !Mode:: "TeX:UTF-8"

\documentclass{resume}
\usepackage{zh_CN-Adobefonts_external} % Simplified Chinese Support using external fonts (./fonts/zh_CN-Adobe/)
%\usepackage{zh_CN-Adobefonts_internal} % Simplified Chinese Support using system fonts
\usepackage{linespacing_fix} % disable extra space before next section
\usepackage{cite}
\usepackage{hyperref}

\renewcommand{\normalsize}{\small} 

\begin{document}
\pagenumbering{gobble} % suppress displaying page number

\name{丁鼎 Eric Ding}

% {E-mail}{mobilephone}{homepage}
% be careful of _ in emaill address
\contactInfo{(+86) 13601629872}{ericding@umich.edu}{\href{https://ericdinging.github.io/mywebsite/}{https://ericdinging.github.io/mywebsite/}}{}
% {E-mail}{mobilephone}
% keep the last empty braces!
%\contactInfo{xxx@yuanbin.me}{(+86) 131-221-87xxx}{}
 
% \section{个人总结}
% 本人在校成绩优秀、乐观向上,工作负责、自我驱动力强、热爱尝试新事物,认同开放、连接、共享的Web在未来的不可替代性。在校期间长期从事可视分析(Web的2D/3D时空可视化)相关研究,对Web技术发展趋势及前端工程化解决方案有浓厚兴趣。\textbf{现任职于阿里巴巴集团。}

% \section{\faGraduationCap\ 教育背景}
\section{教育背景}
\datedsubsection{\textbf{康奈尔大学},计算机科学,\textit{博士研究生}}{2024.9 - 2029.6}
\ 导师: Rachee Singh, 研究方向: 机器学习系统与网络, 康奈尔大学研究生奖学金 (前20\%)
\datedsubsection{\textbf{密西根大学},计算机科学,\textit{工学学士}}{2022.8 - 2024.5}
\ \textbf{GPA. 3.96}, 导师: Mosharaf Chowdhury, James B. Angell Scholar, Dean's List
\datedsubsection{\textbf{上海交通大学},电子信息与计算机工程,\textit{工学学士}}{2020.9 - 2024.8}
\ \textbf{GPA. 3.76 (前10\%)}, 唐君远奖学金提名

% \section{\faCogs\ IT 技能}
\section{技术能力}
% increase linespacing [parsep=0.5ex]
\begin{itemize}[parsep=0.2ex]
  \item \textbf{编程语言}: C++, Python, C, SystemVerilog, RISC-V, Matlab, React/JS, Bash, SQL, R
  \item \textbf{框架与协议}: PyTorch, CUDA, Tensorflow, TCP/IP, HTTP, gRPC, Hadoop
  \item \textbf{操作系统,数据库与工程构建}: Linux, macOS, Docker, Kubernetes, Redis, MySQL, Git, Cloudflare, AWS
  \item \textbf{工程模拟}: Verdi, Catia, Matlab, Mathematica, LabVIEW, Pspice, Proteus, Vivado  
\end{itemize}

% \end{itemize}

\section{科研经历}
\datedsubsection{\textbf{\href{https://symbioticlab.org/}{SymbioticLab}, 密西根大学}, 机器学习系统研究助理,导师:Mosharaf Chowdhury}{2023.5 - 2024.5}
\begin{itemize}
  \item 开发基于微服务架构的联邦学习(FL)资源管理系统\emph{Propius}。
  \item 在GPU集群中设计和部署分布式FL评估框架,支持多任务并行训练。
  \item 在\emph{Propius}部署高性能调度算法\emph{Venn},将平均FL任务完成时间提高1.88倍,相较于随机分配策略。
  \item 参与开源项目\emph{FedScale},完善FL优化器(FedYoGi, FedProx, FedAvg)部署。
\end{itemize}

\datedsubsection{\textbf{\href{https://fanlab.bme.umich.edu/}{The Fan Lab}, 密西根大学}, 嵌入式系统研究助理, 导师: Xudong Fan}{2023.5-2023.9}
\begin{itemize}
  \item 开发\emph{WASP}无线可穿戴设备,能够通过监测分析汗液蒸发来实现早期疾病诊断。
  \item 设计了基于I2C和Bluetooth Low Energy协议的高可靠性通信协议。构造局部网络,实现两个设备内微控制器与一个终端微控制器之间的高保真数据通信。
\end{itemize}

\datedsubsection{\textbf{\href{https://jhc.sjtu.edu.cn/}{约翰·霍普克罗夫特计算机科学中心}}, 机器学习理论研究助理, 导师: Shuai Li}{2021.9-2022.3}
\begin{itemize}
  \item 研究、理论分析并部署非对比自监督学习(SSL)算法。
  \item 比较分析非对比SSL和传统监督学习的性能,验证了长尾分布的不平衡数据集上SSL方法的稳定性。
\end{itemize}

% \begin{onehalfspacing}
% \end{onehalfspacing}

% \datedsubsection{\textbf{DID-ACTE} 荷兰莱顿}{2015年3月 - 2015年6月}
% \role{本科毕业设计}{LIACS 交换生}
% 利用结巴分词对中国古文进行分词与词性标注,用已有领域知识训练形成 classifier 并对结果进行调优
% \begin{onehalfspacing}
% \begin{itemize}
%   \item 利用结巴分词对中国古文进行分词与词性标注
%   \item 利用已有领域知识训练形成 classifier, 并用分词结果进行测试反馈
%   \item 尝试不同规则,对 classifier 进行调优
% \end{itemize}
% \end{onehalfspacing}

\section{项目经历}
% increase linespacing [parsep=0.5ex]
\begin{itemize}[parsep=0.2ex]
  \item 用SystemVerilog编写完整的\textbf{可综合乱序RICS-V处理器},功能包括:R10k寄存器重命名算法,N路超标量,加载/存储队列,锦标赛分支预测器和非阻塞缓存。
  \item 编写\textbf{前向卷积层计算CUDA核函数},采用了内存协同、共享内存乘法和循环展开等高性能并行技术。
  \item 用Python/React-JS编写\textbf{全栈分布式搜索引擎}。构造兼容Hadoop Streaming的MapReduce Pipeline,计算Wikipedia网页的分段倒排索引。
        编写了一个分布式后端索引服务,能够通过PageRank和TF-IDF生成定制搜索结果,并编写了可扩展的前端搜索服务器。该项目在AWS上部署。
  \item 用C++编写\textbf{简化操作系统内核},开发了CPU调度器和线程库,支持线程分配、中断和同步API,如互斥锁和条件变量。开发了一个内存页面调度器,管理进程的虚拟地址空间和磁盘文件缓存。用UNIX文件系统层次结构构建了一个多线程网络文件服务器。
\end{itemize}

\section{发表文章}
\begin{itemize}[parsep=0.2ex]
  \item Jiachen Liu, Fan Lai, \textbf{Ding Ding}, Yiwen Zhang, Mosharaf Chowdhury, \href{https://arxiv.org/abs/2312.08298}{Venn: Resource Management Across Federated Learning Jobs}, arXiv (arXiv:2312.08298)
  \item Anjali Devi Sivakumar, Ruchi Sharma, Chandrakalavathi Thota, \textbf{Ding Ding}, Xudong Fan, \href{https://pubs.acs.org/doi/10.1021/acssensors.3c01936}{WASP: Wearable Analytical Skin Probe for Dynamic Monitoring of Transepidermal Water Loss}, ACS Sensors 2023.
\end{itemize}
% \section{\faHeartO\ 项目/作品摘要}
% \section{项目/作品摘要}
% \datedline{\textit{An Integrated Version of Security Monitor Vis System}, https://hijiangtao.github.io/ss-vis-component/ }{}
% \datedline{\textit{Dark-Tech}, https://github.com/hijiangtao/dark-tech/ }{}
% \datedline{\textit{融合社交网络数据挖掘的电视节目可视化分析系统}, https://hijiangtao.github.io/variety-show-hot-spot-vis/}{}
% \datedline{\textit{LeetCodeOJ Solutions}, https://github.com/hijiangtao/LeetCodeOJ}{}
% \datedline{\textit{Info-Vis}, https://github.com/ISCAS-VIS/infovis-ucas}{}


% \section{\faInfo\ 社会实践/其他}
\section{实践及其他}
% increase linespacing [parsep=0.5ex]
\begin{itemize}[parsep=0.2ex]
  \item Computer Science Peer Tutor,密西根大学Renew CS Tutoring Program, 2024.2 - 2024.5
  \item 电气工程师,上海交通大学赛车队,2021.3 - 2022.8
  \item 班长,上海交通大学密西根学院,2020.9 - 2022.8
\end{itemize}

%% Reference
%\newpage
%\bibliographystyle{IEEETran}
%\bibliography{mycite}
\end{document}
